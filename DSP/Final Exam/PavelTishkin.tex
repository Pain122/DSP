\documentclass{article}
\usepackage[utf8]{inputenc}
\usepackage{amsmath}
\usepackage{nicefrac}

\title{Pavel Tishkin, 30.04.2000, DS18-01}
\author{DSP Midterm 1}
\date{May 2021}

\usepackage[left=36mm, right=36mm, top=0.4in, bottom=0.5in]{geometry}

\begin{document}

\maketitle
\section{Task 1}
\textbf{Problem:}
Design (according to your actual variant) a filter to process signals with a period of length $m=year$ that passes (without any change) frequencies $\frac{2k\pi}{m}$ for all $k \in [0..(m-1)]$ but $k=day$ and $k=month$. Explain the design algorithms and all design steps (providing references to the properties justifying the steps)!
\newline
\textbf{Solution:}
\newline
The algorithm to design the filter may be written, as follows:
\begin{itemize}
    \item \textbf{Step 1)} Identify the necessary frequencies in the frequencies domain
    \item \textbf{Step 2)} Transform the result from the previous step to the impulse domain
\end{itemize}
\newline
\newline
\textbf{Step 1)} Identify the necessary frequencies in the frequencies domain
According to my variant, the necessary frequencies are:
$$\frac{2k\pi}{m}, k\in [0, ... 3, 5, 6 ... 29, 31, 32, ... 1999]$$
This can be represented as a signal of length $m$
\newline
\newline
\textbf{Step 2)} Transform the result from the previous step to the impulse domain.
\newline
Since the filter used is going to be applied to periodic signal and we need discrete frequencies, we will use IDFT to build the filter. Firstly, let us list the necessary tools for filter design:
\begin{itemize}
    \item Kronecker Delta sequence
    \begin{equation}
    \label{eqn:delta}
        \delta = (\mathbf{1}, 0, ..., 0)
    \end{equation}
    \item Kronecker Delta sequence DFT
    \begin{equation}
    \label{eqn:deltaDFT}
        \delta_n \stackrel{\mathrm{DFT}}{\longleftrightarrow} 1
    \end{equation}
    \item Constant sequence
    \begin{equation}
    \label{eqn:const}
        1 \stackrel{\mathrm{DFT}}{\longleftrightarrow} m\delta_k \text{, where $m$ - sequence lenght, or $2000$ in our case}
    \end{equation}
\end{itemize}
We also need some properties of DFT:
\begin{itemize}
    \item Linearity
    \begin{equation}
    \label{eqn:lin}
        \alpha x_n + \beta y_n \stackrel{\mathrm{DFT}}{\longleftrightarrow} \alpha X_k + \beta Y_k
    \end{equation}
    \item Circular frequency shift or modulation
    \begin{equation}
    \label{eqn:shift}
        W_m^{-np}x_n \stackrel{\mathrm{DFT}}{\longleftrightarrow} X_{(k-p)mod\ m}\text{, where }W_m = e^{\nicefrac{-2j\pi}{m}}
    \end{equation}
\end{itemize}
Now, we are ready to build the filter itself. Let $Y_k = 1,\ Z_k = \delta_{k-4},\ H_k = \delta_{k-30}$ Then, we can represent it in frequency domain as a sum of three filters:
$$X_k = Y_k - Z_k - H_k$$
Using Linearity propety [\ref{eqn:lin}], our result can be represented as:
$$x_n = y_n - z_n - h_n \stackrel{\mathrm{DFT}}{\longleftrightarrow} Y_k - Z_k - H_k = X_k$$
Let us compute IDFT of sequences $Y, Z, H$:
\begin{itemize}
    \item Sequence $Y$: Its IDFT result is a standard Kronecker Delta sequence as presented in [\ref{eqn:deltaDFT}]:
    $$Y_k = 1 \stackrel{\mathrm{IDFT}}{\longleftrightarrow} \delta_n = y_k$$
    \item Sequence $Z$: To get IDFT of this sequence to be as [\ref{eqn:const}], we need to multiply and divide it by $m=2000$ (Using linearity property \ref{eqn:lin}) and shift it by $4$ [\ref{eqn:shift}]:
    $$Z_k = \frac{1}{2000}\delta_{(k-4)mod\ 2000} \stackrel{\mathrm{IDFT}}{\longleftrightarrow} \frac{1}{2000} W_m^{-4n} \cdot 1 = \frac{e^{\nicefrac{jn\pi}{250}}}{2000}$$
    \item Sequence $H$: To get IDFT of this sequence to be as [\ref{eqn:const}], we need to multiply and divide it by $m=2000$ (Using linearity property \ref{eqn:lin}) and shift it by $30$ [\ref{eqn:shift}]:
    $$Z_k = \frac{1}{2000}\delta_{(k-30)mod\ 2000} \stackrel{\mathrm{IDFT}}{\longleftrightarrow} \frac{1}{2000} W_m^{-30n} \cdot 1 = \frac{e^{\nicefrac{3jn\pi}{100}}}{2000}$$
\end{itemize}
In the end, we get:
$$
x_n = y_n - z_n - h_n = \mathbf{\delta_n - \frac{e^{\nicefrac{jn\pi}{250}}}{2000} - \frac{e^{\nicefrac{3jn\pi}{100}}}{2000}} \stackrel{\mathrm{DFT}}{\longleftrightarrow} \mathbf{X_k}
$$
\textbf{Answer:}
$$
x_n = \delta_n - \frac{e^{\nicefrac{jn\pi}{250}}}{2000} - \frac{e^{\nicefrac{3jn\pi}{100}}}{2000} \stackrel{\mathrm{DFT}}{\longleftrightarrow} X_k
$$
\section{Task 2}
\textbf{Problem:}
Design (according to your actual variant) an ideal low-pass filter to process infinite discrete signals that passes (without any change) only frequencies in range $[-\frac{day}{year}, +\frac{month}{year}]$. Explain the design algorithms and all design steps (providing references to the properties justifying the steps)!
\newline
\textbf{Solution:}
The algorithm to design the filter may be written, as follows:
\begin{itemize}
    \item \textbf{Step 1)} Identify the necessary frequencies in the frequencies domain
    \item \textbf{Step 2)} Transform the result from the previous step to the impulse domain
\end{itemize}
Let us proceed with the calculations of the filter, using the algorithm:
\newline
\newline
\textbf{Step 1)} Identify the necessary frequencies in the frequencies domain.
\newline
Fist of all, let us put the values of my variant to the range of frequences allowed by the low-pass filter:
$$\omega \in [-\frac{30}{2000}, +\frac{4}{2000}]$$
We will simplify the fractions once we achieve the result for simplicity of some calculations.
The result can be presented as a following sequence:
$$
X(e^{j \omega}) = \text{ if } \nicefrac{-30}{2000} \le \omega \le +\nicefrac{4}{2000} \text{ then } 1 \text{ else } 0 
$$
\newline
\newline
\textbf{Step 2)} Transform the result from the previous step to the impulse domain.
\newline
Cardinal sine signal used as the low-pass filter is:
\begin{equation}
\label{eqn:sinc}
\sqrt{\frac{\omega_0}{2 \pi}} sinc \frac{\omega_0 n}{2} \stackrel{\mathrm{DTFT}}{\longleftrightarrow} 
\begin{cases}
\sqrt{\nicefrac{2\pi}{\omega_0}} \text{, if } |\omega|\le \frac{\omega_0}{2} \\
0 \text{, otherwise}
\end{cases}
\end{equation}
To build a filter suitable for our task, we may also require some properties of Discrete Time Fourier Transform:
\begin{equation}
\label{eqn:prop1}
    \alpha x_n + \beta y_n \stackrel{\mathrm{DTFT}}{\longleftrightarrow} \alpha X(e^{j\omega}) + \beta Y(e^{j \omega})
\end{equation}
\begin{equation}
\label{eqn:prop2}
    x_{-n} \stackrel{\mathrm{DTFT}}{\longleftrightarrow} X(e^{-j\omega})
\end{equation}
\begin{equation}
\label{eqn:prop3}
    x_{n-m} \stackrel{\mathrm{DTFT}}{\longleftrightarrow} e^{-j\omega m} X(e^{j\omega}) \text{ and } e^{j \phi n} x_{n} \stackrel{\mathrm{DTFT}}{\longleftrightarrow} X(e^{j(\omega - \phi)}) 
\end{equation}
We need a filter signal such that:
$$
X(e^{j \omega}) = \text{ if } \nicefrac{-30}{2000} \le \omega \le +\nicefrac{4}{2000} \text{ then } 1 \text{ else } 0 
$$
First of all, let us find suitable $\omega_0$. First of all, we would need to shift the signal by $\phi = \nicefrac{-13}{2000}$:
$$
\text{ if } \nicefrac{-30}{2000} - \nicefrac{-13}{2000} \le \omega - \nicefrac{-13}{2000} \le +\nicefrac{4}{2000} - \nicefrac{-13}{2000} \text{ then } 1 \text{ else } 0 = 
$$
$$
= \text{ if } \nicefrac{-17}{2000} \le \omega + \nicefrac{13}{2000} \le +\nicefrac{17}{2000} \text{ then } 1 \text{ else } 0 
$$
Therefore, $\frac{\omega_0}{2} = \nicefrac{17}{2000}$, $\omega = \nicefrac{34}{2000}$. Let us multiply and divide the function by $\phi = \sqrt{\frac{2\pi}{\nicefrac{34}{2000}}}$:
$$\frac{1}{\sqrt{\frac{2\pi}{\nicefrac{34}{2000}}}}(\text{ if } \nicefrac{-17}{2000} \le \omega + \nicefrac{13}{2000} \le +\nicefrac{17}{2000} \text{ then } \sqrt{\frac{2\pi}{\nicefrac{34}{2000}}} \text{ else } 0)$$
From this, it is immediately obvious we got the Cardinal sine signal from [\ref{eqn:sinc}], but it is shifted by $\nicefrac{-13}{2000}$ and is multiplied by $\frac{1}{\sqrt{\frac{2\pi}{\nicefrac{34}{2000}}}}$. Using properties [\ref{eqn:prop1}] for handling constant and [\ref{eqn:prop3}] for handling shift, we get the Inverse Discrete Time Fourier Transform of the filter:
$$\sqrt{\frac{\nicefrac{34}{2000}}{2\pi}} \cdot \sqrt{\frac{\nicefrac{34}{2000}}{2\pi}} \cdot e^{-j\nicefrac{13}{2000} \cdot n} sinc \frac{\nicefrac{34}{2000} \cdot n}{2} = \mathbf{ \frac{\nicefrac{\mathbf{17}}{1000}}{2\pi} e^{-j\nicefrac{\mathbf{13}}{2000} \cdot n} sinc \frac{\nicefrac{\mathbf{17}}{1000} \cdot n}{2}}$$
\textbf{Answer:}
$$
x_n =  \frac{\nicefrac{17}{1000}}{2\pi} e^{-j\nicefrac{13}{2000} \cdot n} sinc \frac{\nicefrac{17}{1000} \cdot n}{2}
$$
\end{document}
